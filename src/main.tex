%! Author = Marjan
%! Date = 15/04/2025

% Preamble
\documentclass[11pt]{book}

% Packages
\usepackage{amsmath}    % For math symbols and equations
\usepackage{graphicx}   % For including graphics
\usepackage{geometry}   % For adjusting page layout
\usepackage{fancyhdr}   % For custom headers/footers
\usepackage{hyperref}   % For hyperlinks
\usepackage{listings}   % For code listing
\usepackage{lipsum}     % For placeholder text (to test layout)
\usepackage{helvet}
\usepackage{arydshln}
\usepackage{tikz}        % For creating diagrams and drawing
\usepackage{pgfplots}    % For plotting graphs
\usepackage{wasysym}
\usepackage{amstex} % Helvetica as a sans-serif font alternative
\renewcommand{\rmdefault}{phv} % Set the default font family to Helvetica

% Set page margins (A4 paper)
\geometry{top=1in, bottom=1in, left=1in, right=1in}

% Set up the header
\pagestyle{fancy}
\fancyhf{}
\fancyhead[L]{AWS Machine Learning Speciality Notes}
\fancyhead[C]{Your Name}
\fancyhead[R]{\thepage}
% Document
\begin{document}
% Title Page
    \begin{titlepage}
        \centering
        \vspace*{2in}
        \Huge \textbf{AWS Machine Learning Speciality Notes}
        \vfill
        \Large Your Name
        \vfill
        \Large Date: \today
    \end{titlepage}

    \newpage

    \tableofcontents
    \newpage


    \chapter{Introduction}


    \chapter{Data Engineering}


    \chapter{Exploratory Data Analysis}

    \subsection{Python stuff}

    \subsection{Types of Data}

    \begin{itemize}
        \item Numerical
        \item Categorical
        \item Ordinal
    \end{itemize}

    \subsubsection{Numerical}
    - Represents some sort of quantitative measurement - Heights of people, page load times, stock prices etc
    - Discrete Data - Integer based; often counts of some event
    - Continuous Data - Has an infinite number of possible values

    \subsubsection{Categorical}
    - Qualitative data that has no inherent mathematical meaning
    - Can assign number to categories in order to represent them more compactly, but the numbers don't have a mathematical meaning.

    \subsubsection{Ordinal}
    - A mixture of numerical and categorical
    - Categorical data that has mathematical meaning

    \subsubsection{Data Distributions}


% Example Normal Distribution: Wealth in the UK by Age (Dummy Example with Higher Peak)

    \begin{tikzpicture}
        \begin{axis}
            [
            width=\textwidth,   % Set plot width to match text width
            height=7cm,        % Optional height adjustment
            xlabel={Age},      % X-axis label
            ylabel={Wealth (in £)}, % Y-axis label
            grid=both,         % Draw grid
            grid style={dashed, gray!30}, % Style grid lines
            xtick={20,30,40,50,60,70,80}, % Define x-axis ticks
            ymin=0, ymax=0.1,  % Y-axis range (increased to make the peak higher)
            samples=100        % Number of samples for smoothness
            ]

            \addplot[very thick, blue, domain=20:80]
            {1/(8*sqrt(2*pi)) * exp(-0.5*((x-50)/15)^2)};
            \addlegendentry{Dummy Wealth Distribution by Age}
        \end{axis}
    \end{tikzpicture}
    
% Comments:
% - This is a dummy example of a wealth distribution in the UK by age.
% - The curve assumes a mean age of 50 (highest wealth concentration) 
%   and a standard deviation of 15 years.
% - To increase the peak, the denominator in the normal distribution equation was adjusted.
% - This does not represent real-world data and is purely illustrative.


% Bernoulli Distribution Example
    \begin{tikzpicture}
        \begin{axis}
            [
            ybar,
            bar width=15pt,
            ymin=0,
            ymax=1.2,
            xtick={0,1},
            xticklabels={$0$, $1$},
            xlabel={Outcome (Success or Failure)},
            ylabel={Probability},
            grid=both,
            grid style={dashed, gray!30},
            width=10cm,
            height=6cm,
            nodes near coords,
            enlargelimits=0.15,
            symbolic x coords={0,1}
            ]
            \addplot[ybar, fill=blue] coordinates {
                (0,0.4) (1,0.6)
            };
            \addlegendentry{Bernoulli Distribution with $P(X=1) = 0.6$}
        \end{axis}
    \end{tikzpicture}

% Comments:
% - This plot represents an example Bernoulli distribution.
% - P(X=1) = 0.6 means probability of success is 0.6, and thus P(X=0) = 0.4.
% - Bars indicate probabilities of both outcomes.
    \subsection{Time series analysis}

    \subsection{Time Series Analysis: GDP in the UK (Fake Data)}

% Example Fake GDP Time Series Chart Over 50 Years
    \begin{tikzpicture}
        \begin{axis}
            [
            width=\textwidth,
            height=7cm,
            xlabel={Year},
            ylabel={GDP (in Trillions of £)},
            grid=both,
            grid style={dashed, gray!30},
            ymin=0,
            xtick={1975, 1985, 1995, 2005, 2015, 2025},
            legend pos=north west,
            title={\textbf{UK GDP Growth Over 50 Years (Fake Data)}}
            ]
            \addplot[
                thick,
                color=blue,
                mark=*,
                samples at={1975, 1980, 1990, 2000, 2010, 2020, 2025}
            ] coordinates {
                (1975, 0.5) (1980, 0.7) (1990, 1.2) (2000, 1.8) (2010, 2.3) (2020, 2.7) (2025, 3.2)
            };
            \addlegendentry{Fake GDP Data}
        \end{axis}
    \end{tikzpicture}

% Comments:
% - This is a fake time series example showing GDP in the UK over 50 years.
% - The data is not real and serves only for illustrative purposes.
% - The chart represents a hypothetical increase in GDP from 1975 to 2025.


% Example Random Walk Chart
    \begin{tikzpicture}
        \begin{axis}
        [
            width=\textwidth,
            height=8cm,
            xlabel={Step},
            ylabel={Value},
            grid=both,
            grid style={dashed, gray!30},
            ymin=-10, ymax=10,
            xmin=0, xmax=100,
            title={\textbf{Random Walk Example}},
            legend pos=south east,
        ]
            % Simulated random walk values
            \addplot[
                thick,
                color=cyan,
                mark=none
            ] coordinates {
                (0, 0) (1, 0.6) (2, -0.5) (3, 1.2) (4, 0.8) (5, 1.4) (6, 2.1)
                (7, 1.7) (8, 0.9) (9, 1.5) (10, 2.3) (11, 1.6) (12, 2.8)
                (13, 2.2) (14, 1.9) (15, 2.6) (16, 1.8) (17, 2.0) (18, 2.7)
                (19, 3.5) (20, 4.1) (21, 3.2) (22, 2.5) (23, 1.9) (24, 2.4)
                (25, 2.6) (26, 3.9) (27, 3.2) (28, 2.8) (29, 3.0) (30, 2.7)
                (31, 3.5) (32, 2.8) (33, 1.6) (34, 0.8) (35, 1.5) (36, 1.2)
                (37, 0.5) (38, 0.9) (39, -0.3) (40, -1.0) (41, -1.5) (42, -0.8)
                (43, 0.2) (44, -0.1) (45, -0.6) (46, -1.5) (47, -2.8) (48, -3.0)
                (49, -4.2) (50, -3.8) (51, -3.0) (52, -3.6) (53, -2.9) (54, -2.5)
                (55, -2.7) (56, -2.0) (57, -3.1) (58, -3.8) (59, -4.0) (60, -3.2)
                (61, -2.9) (62, -1.6) (63, -2.2) (64, -1.8) (65, -0.4) (66, 0.2)
                (67, 0.0) (68, -0.8) (69, -0.4) (70, 0.1) (71, 1.2) (72, 0.7)
                (73, -0.5) (74, -0.1) (75, -0.6) (76, -1.2) (77, -0.8) (78, -0.3)
                (79, -0.5) (80, 0.2) (81, 1.3) (82, 0.9) (83, 1.2) (84, 0.5)
                (85, 1.8) (86, 2.0) (87, 2.5) (88, 2.9) (89, 3.5) (90, 3.2)
                (91, 2.5) (92, 2.7) (93, 3.0) (94, 3.2) (95, 3.7) (96, 2.9)
                (97, 2.2) (98, 1.8) (99, 2.0) (100, 2.5)
            };
            \addlegendentry{Random Walk}
        \end{axis}
    \end{tikzpicture}

    %TODO ------ revisit this topic
    \subsection{Introduction to Amazon Athena}
    \subsection{Overview of Amazon Quicksight}
    \subsection{Types of visualisations and when to use them}
    \subsection{Elastic MapReduce (EMR) amd Hadoop Overview}
    \subsection{Apache Spark on EMR}
    \subsection{Feature Engineering and the Curse of Dimensionality}
    \subsection{Inputing Missing Data}
    \subsection{Dealing with Unbalanced Data}
    \subsection{Handling Outliers}
    \subsection{Binning, Transforming, Encoding, Scaling and Shuffling}
    \subsection{Amazon SageMaker Ground Truth and Label Generation}
    \subsection{Lab 1}
    \subsection{Lab 2}

    \chapter{Modeling, Part 1: General Deep Learning and Machine Learning}


    \chapter{Modeling, Part 2: Amazon SageMaker}


    \chapter{Modeling, Part 3: High-Level ML Services}


    \chapter{Modeling, Part 4: Wrapping up and Labs}


    \chapter{ML Implementation and Operationsl}


    \chapter{Practice Exam Questions}


    \chapter{Generative AI: Transformers GPT, Self-Attention and Foundation Models}


    \chapter{Wrapping Up}

\end{document}